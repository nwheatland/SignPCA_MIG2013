
\section{Overview}

Our overall technique is divided into two stages: 1) the computation  
of the sparse marker dataset; and 2) reconstruction from the sparse 
marker set to full-resultion, skeleton-driven hand animation.
To start, we collect full-resolution motion capture 
data of hand in a small capture region. 
Specifically, our actor wears 13 small
(approx. 6mm) markers directly on the hands as well as three  
markers on the lower forearm. 
The lower forearm acts as the root link for our hand skeleton with the
assumption that these same three markers will appear in full-body captures. 
%The recorded captures act as our database. 
To account for gross body hand motion, marker positions in the database are put into the same coordinate
frame by computing the transformation of each marker relative to  
the root link.

To complete the first phase, we employ the reference data and perform PCA over the markers and 
derive a rank ordering for the markers based on their influences over
the principle components.   From this rank-ordered list, we can easily
select the top markers to act as our sparse marker set.
%
For the second phase, reconstruction, we set up a locally weighted regression 
(LWR) model to map from the sparse marker positions to a set of derived principle
components.  In this case, we conduct the PCA from the joint angles.  The LWR model is built
for each test query based on the input markers for the query sample and their proximity to 
the analogous markers in the reference data after correcting for lower arm movement.
%
Joint angles for the low dimensional input is then reconstructed 
by reversing the PCA process, from the principle components produced by 
the regression back to a full set of joint angles. 
%The joint angles are then fed back into the mapping 
%program to animate the hand.
