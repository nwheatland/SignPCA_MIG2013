

\section{Results}
Our primary ``ASL'' database is composed simply of two contiunous
runs of the signs of the letters for the complete 
alphabet,
signed by the same motion subject. 
The sign language sequences that we  
reconstruct include various words in sign language, including
signs not in the database.

For our small set of markers, we choose to use six markers as our
baseline in order to compare our technique to existing solutions. 
Using our method to determine 
marker importance, the markers chosen are all of the fingertips
and one on the lower part of the index finger. We also choose a
marker set of three. The markers chosen are the fingertips of the 
middle, ring, and pinky fingers.\\
%
We compare our marker set of six to marker sets presented by
Kang et al.(2012) and Hoyet et al. (2011). The differences can
be seen in video form. Using the regression method, our marker 
set produces a smaller average joint angle error per frame 
for all of our current sign language tests. Hoyet's method 
consistently has the largest joint angle error.\\