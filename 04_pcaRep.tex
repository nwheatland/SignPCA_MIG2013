\section{PCA representation for hands}

At the core of our technique is the assumption that hand motion is
relatively low-dimensional.  Even though a full resolution skeleton
of the hand can have upwards for one hundred degrees of freedom (DOF),
many of the DOFs of the hand are dependent \cite{SanFlaSoe98,BraZha04,JoeOSu09}. In our approach, we exploit
this low-dimensional assumption through PCA, anticipating that PCA
will hold the bulk of the important features of the whole-body hand
motion in a relatively small number of principle components.  We
support this primary assumption by showing that joint angles alone
are not capable of producing... 

We experimented with two ways to construct 
the joint angles from the 
marker database.
Principle component analysis (PCA) is performed 
over the joint angles, giving us a joint to principle component 
representation with 54 components (3 for each Euler angle) 
for every sample of motion in the database.  
%
PCA is performed on the marker positions of the 
database, giving us 39 components (3 for each XYZ position) for every 
sample in the database. PCA identifies the most important information 
in this database by making it the first component. Each subsequent 
component is less valuable than its predecessor. Using this knowledge, 
we sum up the values of each component over all of the samples and 
weigh each sum based on importance (which component). Each weighted 
sum correlates with a marker, and the markers are then ordered in terms 
of importance by their weighted sum. The chosen marker set is then 
used for the data we wish to reconstruct. \\

To reconstruct a sequence of motion, we first only use data from the 
marker set selected in the previous step. This is the equivalent of 
having our performer only wear these markers in the initial capture 
session. These marker position are plugged into our regression model 
to determine the full motion of our low dimensional sequence in 
principle component space. \\

Joint angles for the low dimensional motion is then reconstructed 
by reversing the PCA process on the principle components produced by 
the regression. The joint angles are then fed back into the mapping 
program to animate the hand.