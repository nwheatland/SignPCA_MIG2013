

\section{Introduction}
Producing quality whole-body motion involves the movement of the hand in relation to the rest of the body. Using a motion capture system, it is difficult to record the full body of a moving person while also capturing the hand and all of its detail because the whole-body and hand appear at largely different scales.  While it is possible to record a high-resolution capture of the hand through a comprehensive set of markers (typically 13-20 markers), this is best conducted in a small capture region, isolating the motion of the hand.  However, in a larger, full-body capture region the complete set of markers becomes difficult to discern and so this approach is usually abandoned in lieu of the capture of a smaller set of markers (2-6 markers) coupled with a process for reconstructing the full hand animation.  In this paper, we provide a robust technique for the latter that both automatically selects a sparse marker set to record and subsequently produces joint trajectories for a full skeleton from the sparse marker set.  

Specifically, our technique employs a combination of Principle Component Analysis (PCA)~\cite{blahPCA} to construct a low-dimensional representation of the hand data along with a linearly weighted regression (LWR) model to aid in the reconstruction step.  Starting from a reference database that is recorded using a full-resolution marker set, we both determine the best sparse marker set based on the PCA representation from this data as well as use it in the synthesis step with LWR.  We experimented the size of the marker set to record, specifically reduced marker sets of six and three markers and compare our technique with different approaches proposed for selecting the marker, including manual selection as in ~\cite{HoyRyaOSu11} and the use of an representative cluster-based search approach~\cite{KanWheZor12}. In contrast, our technique computes the marker set directly.

For reconstruction, our method employs a smaller number of markers to generate full-hand motion. Based on a test query, we use LWR to build a local model of the full-resolution PCA-version... The results from our reconstructions show that much of the finger specificity needed when communicating with sign language is preserved. We show the power of our technique using American Sign Language (ASL) as a primary testbed along with a few other more generic hand motions, including gesturing and finger-counting.