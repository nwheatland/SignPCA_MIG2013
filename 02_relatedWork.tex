
\section{Related work}

The detailed and subtle motions of the hands are hard to capture. Several approaches for recording have been suggested, each of them having advantages and disadvantages. In particular, optical motion capture systems, while being very accurate, can require substantial post-processing to handle occlusions and mislabelings. While Cyber Gloves~\shortcite{Cyb13} can deal with captures in larger spaces, they also require regular calibrations and do not provide high enough accuracy for many applications~\cite{KahZacKle04}. For other camera-based or range-scan type systems, the hand needs to be in a confined space and the body can not captured synchronously~\cite{WanPop09,ZhaChaXu12}.   The lack of robust recording solutions has lead to a practice of
hand-over animation in industry, foregoing detail capture in lieu of often laborious post-processing.
% the citation is about cyberglove 2, but I am citing cyberglove 3 previously

Our aim focuses on facilitating the quality capture of hand motions, together with full-body motions, in a motion capture system. To reach this goal, we investigate the most effective way to capture accurate hand motions using the smallest possible number of markers and suggest a corresponding, specialized hand-over technique to reconstruct the full hand from the markers. Other researchers have analyzed finger motions and found strong correlations between different degrees of freedom. Rijpkema and Girard \shortcite{RijGir91} report that the relationship between the flexion of the distal and the proximal interphalangeal joint (DIP and PIP, respectively) is approximately linear, with $DIP=2/3*PIP$. J\"{o}rg and O'Sullivan \shortcite{JoeOSu09} show how to reduce the degrees of freedom of the hands by eliminating irrelevant and redundant information. These approaches reveal that finger motion is highly redundant. And we take advantage of  correlation between different degrees of freedom of the hand to optimize the capturing and construction of quality hand animation.

Principal component analysis, as a standard technique to analyze and reduce high-dimensional data, has also been used to study hand movement. Braido and Zhang~\shortcite{BraZha04}, show that for the hand the two first principal components of a PCA accounted for over 98\% of all variance in the joint angles. However, their motion database did not take into account the thumb and involved only two types of tasks - cylinder grasping and voluntary flexion of individual fingers. Santello et al. \shortcite{SanFlaSoe98} studied a variety of grasp poses and found that over 80\% of the measured degrees of freedom could be described by their first two principal components. 
%Ciocarlie...
However, these studies, applied to grasps, do not require specific motions from individual fingers. In contrast, we present a method applied to American sign language (ASL), which exhibits impressive dexterity and variety of finger motions. We hypothesize that there is less redundancy in typical finger motions of ASL than in standard grasping motions. 

One of our goals is to determine which is the most effective set of markers for capturing hand motions. Previous work has studied best marker sets for hand motions, for example,  by testing and comparing markers sets chosen manually (with reconstruction done using inverse kinematics)~\cite{HoyRyaOSu11}, or through a brute-force approach, that compares the error of similar poses found in a database~\cite{KanWheZor13}.  Chai and Hodgins~\shortcite{ChaiLow} studied full-body motion with similar goals to the ones described in this paper, but once again use a manually selected marker set.  

%  Would be good to pull in Sophie's work from last year as a point of reference...

%Hoyet...: They find that for many cases a simple 8 marker hand model - one on each fingertip, three of the bases of the thumb, index, and little finger, produces good results. 
%Kang et al. : the best set of six markers includes the fingertips of the index, middle, ring finger as well as three markers on or near the thumb. 
%Sign language: ...
%Hand low-res: KanWheZor12c, CioGolAll07, ChaPolXin07

%Correlations between body and finger motions, in case we manage to include data on the body: MajZorFal06, JoeHodSaf12

%Body low-res?: SafHodPol04
%Perception?: JoeHodOSu10
