
\section{Reconstruction}

A small number of markers alone cannot produce a fully
realized hand animation. Therefore, we build a regression model 
to construct joint angle measurements for a full motion sequence.
Our LWR model maps marker positions in our reference database
to joint angles of our database represented in principle
component space.

For this step, we introduce an input motion sequence
that uses the sparse marker set. The input data is 
plugged into the regression model to predict the best
joint angle components to be used for final reconstruction.
Reconstruction is completed when PCA process described in 
Section 4 is reversed, using the newly predicted components
as input. The result is a full set of joint angles for the
input motion sequence.


%We test our method on a variety sign language 
%sequences. 

%RESULTS?
% We determine the quality of these animations
% by comparing the reconstructed joint angles to ground truth
% original data. Our method also employs the principle
% component analysis (PCA), and we can access quality by 
% comparing the components of the reconstructed joint angles
% to the components of original joint angles.


