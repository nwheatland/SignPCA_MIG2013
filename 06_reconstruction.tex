
\section{Reconstruction}

The reconstruction process takes as input a recorded sequence of the sparse marker set.  It produces joint angle trajectories that estimate the full hand motion.  
%A small number of markers alone cannot produce a fully
%realized hand animation. Therefore, 
To this end, we build a regression model to construct joint angle measurements for a full motion sequence. Specifically, our locally weighted regression (LWR) model maps marker positions in the recorded sequence to principle components. Subsequently, the principle components are converted into joint angles using the covariance matrix from the PCA to produce the final motion.

The LWR model is built for each individual sample, or query, taken from the
recorded sequence.  Through this process,
each instance in the database is weighted and this weighting is used
to bias the model.  The weighting is computed as the inverse
of the Euclidean distance from the (root-link corrected) marker positions 
between the query and the samples in the database.  The result
is a regression that places importance on the reference samples
that are close to the test query, while also down-weighting the
the influence of reference samples which are distant
from the query.  
%For further discussion on this topic, we
%refer interested readers to similar efforts with whole-body work~\cite{}.

At run-time, we introduce an input sequence
recorded from the sparse marker set. The input data is 
put through the regression modeling step to predict the principle components.
To ensure smoothness, the trajectories of the principle components
are filtered before they are converted into joint angles.  In our results, 
we use a cone filter with a size of seventeen (with our sample rate for
the motion recordings set at $120~hz$.)

We also experimented with
filtering the joint angles to produce smoothness, but found more visually 
appealing results when we filtered the principle components.  Our 
assumption for this finding is that the principle components combine
to produce more ``crisp'' motion even when they are filtered, while
the joint angle filtering dilutes the unique features of individual
poses over time.  Further study of this phenomena is likely to 
reveal some interesting findings.


%Reconstruction is completed when PCA process described in 
%Section 4 is reversed, using the newly predicted components
%as input. The result is a full set of joint angles for the
%input motion sequence.


%We test our method on a variety sign language 
%sequences. 

%RESULTS?
% We determine the quality of these animations
% by comparing the reconstructed joint angles to ground truth
% original data. Our method also employs the principle
% component analysis (PCA), and we can access quality by 
% comparing the components of the reconstructed joint angles
% to the components of original joint angles.


