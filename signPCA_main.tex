%%% template.tex
%%%
%%% This LaTeX source document can be used as the basis for your technical
%%% paper or abstract. Intentionally stripped of annotation, the parameters
%%% and commands should be adjusted for your particular paper - title, 
%%% author, article DOI, etc.
%%% The accompanying ``template.annotated.tex'' provides copious annotation
%%% for the commands and parameters found in the source document. (The code
%%% is identical in ``template.tex'' and ``template.annotated.tex.'')

\documentclass[review]{acmsiggraph}

\TOGonlineid{45678}
\TOGvolume{0}
\TOGnumber{0}
\TOGarticleDOI{1111111.2222222}
\TOGprojectURL{}
\TOGvideoURL{}
\TOGdataURL{}
\TOGcodeURL{}

\title{Automatic Hand-Over Animation using Principle Component Analysis}

\author{Nkenge Wheatland \hspace{10 mm}  
Sophie J\"{o}rg
\hspace{10 mm} 
Victor Zordan 
\\UC Riverside
\hspace{16 mm} Clemson University
\hspace{16 mm} UC Riverside}
%\thanks{e-mail:rsmith@gmail.com}
\pdfauthor{}

\keywords{character animation, motion capture, hand motion, dimensionality reduction, PCA}

\begin{document}

%% \teaser{
%%   \includegraphics[height=1.5in]{images/sampleteaser}
%%   \caption{Spring Training 2009, Peoria, AZ.}
%% }

\maketitle

\begin{abstract}
This paper introduces a method for producing high quality hand motion using a
small number of markers. The proposed ``hand-over'' animation technique constructs
joint angle trajectories for the hand with the help of a full-resolution reference database. 
Utilizing principle component analysis (PCA) applied to the database, the system automatically
determines the sparse marker set to record. Further, to produce the hand animation,
PCA is used along with a locally weighted regression model to reconstruct joint angles.   
The resulting animation is a full-resolution hand which reflects the original motion without
the need for capturing the full-resolution marker set.  
Comparing the technique to other
methods reveals improvement over the state of the art in terms of the marker set selection.
In addition, the results highlight the ability to generalize the motion synthesized, both
by extending the use of a single reference database to new motions, and from distinct 
reference datasets, over a variety of freehand motions.



\end{abstract}

\begin{CRcatlist}
  \CRcat{I.3.3}{Computer Graphics}{Three-Dimensional Graphics and Realism}{Display Algorithms}
  \CRcat{I.3.7}{Computer Graphics}{Three-Dimensional Graphics and Realism}{Radiosity};
\end{CRcatlist}

\keywordlist

\TOGlinkslist

\copyrightspace


\section{Introduction}

Producing quality whole-body motion involves the movement of the hand in relation to the rest of the
body. However, using a motion capture system, it can be difficult to record the full body of a
moving person while also capturing the hand and all of its detail because the whole-body and hand
appear at largely different scales.  While it is possible to record a high-resolution capture of the
hand through a comprehensive set of markers (typically 13-20 markers), this is often only possible
in a small capture region, isolating the motion of the hand.  However, in a larger, full-body
capture region, the complete set of markers becomes difficult to discern, and so this approach is
usually abandoned in lieu of the capture of a smaller set of markers (2-6 markers) coupled with a
``hand-over'' process for reconstructing the full hand animation~\cite{KanWheZor12}.  In this paper,
we propose a robust technique to accomplish the latter that both automatically selects the marker
set to record, and subsequently produces joint trajectories for a full skeleton from the marker set.
 

Our technique employs a combination of principle component analysis (PCA)~\cite{bishopPCA} to
construct a low-dimensional representation of the hand data along with linearly weighted regression
(LWR) to aid in the reconstruction.  Starting from a reference database that is recorded using a
full-resolution marker set, we first determine the best \emph{sparse} marker set to record based on
a PCA representation of this data.  We experiment with different test sizes for the marker set to
record, specifically reduced marker sets of six and three markers, and we compare our selection
method with different ones proposed for selecting the markers, including manual selection,
following~\cite{HoyRyaOSu11}, and a method that uses representative cluster-based search for
selection~\cite{KanWheZor12}. In contrast, the technique in this paper computes the marker set
directly from the PCA, and our findings show that this marker set is superior to the other methods
of selection for the reconstruction techniques we tested.
For reconstruction, our proposed method employs a second PCA in a synthesis step combined with LWR.  
Starting from a test query that records only the sparse marker set, 
we use LWR to build a locally sensitive model between the markers and the principle components.  These
are then converted back to joint angles to complete the reconstruction of the recorded hand.  

We show the power of our technique using American Sign Language (ASL) as our primary testbed. ASL
is an important and interesting freehand application of hand motion.  Further, it includes a richly diverse
set of configuration poses for the hands.   We show that we can construct new (unseen) ASL signs with 
high-visual quality using a relatively simple, generic ASL database.  Generalization of the database
reveals that we can use our technique to capture other motions, such as counting, even though they
are not closely related to the original motion.  However, this database reveals itself to be too specialized
for subtler freehand motion, such as figurative gestures.  Instead, when a more closely related
gesture reference database appears, novel gesture animation quality improves drastically.   

Our effort holds close similarities to previous work, especially the full-body motion control
of~\cite{Chai05}.  In contrast, our main contributions include the distinct exploration of rich hand
data, such as ASL, as well as our method for determining the best reduced marker set to take
advantage of the power of dimensionality
reduction realized by PCA.  Further, our approach is far simpler and lends itself to ease-of-use and
re-implementation.
Our approach also has notable advantages over other related papers for hand-over animation,
such as~\cite{HoyRyaOSu11,KanWheZor12} in that we compute the best reduced marker set directly,
rather than selecting it manually, or through brute-force search.
Compared to all of these techniques, our technique is both simple to implement and fast to compute,
striking a valuable compromise which is likely to lead to greater adoption for commercial use.

% along with a few other more generic hand motions, including gesturing and finger-counting.

\section{Related work}

The detailed and subtle motions of fingers are hard to capture. Many approaches have been suggested, each of them having their own advantages and disadvantages. Optical motion capture systems, while being very accurate, require substantial post-processing due to occlusions and mislabelings. CyberGloves \cite{Cyb13} require regular calibrations and do not provide the accuracy needed for our purpose \cite{KahZacKle04}. For image-based systems or systems using depth-data, the hand needs to be in a confined space and the body can not captured synchronously \cite{WanPop09,ZhaChaXu12}. 
% the citation is about cyberglove 2, but I am citing cyberglove 3 previously

Our work focuses on facilitating capturing hand and finger motions together with body motions in a motion capture system. To reach this goal, we investigate the most effective way to capture accurate hand motions using the smallest possible number of markers and an effective reconstruction method. Previous research analyzed finger motions to find correlations between different degrees of freedom. Rijpkema and Girard \shortcite{RijGir91} found that the relationship between the flexion of the distal and the proximal interphalangeal joint (DIP and PIP, respectively) is approximately linear with $DIP=2/3*PIP$. J\"{o}rg and O'Sullivan \shortcite{JoeOSu09} reduce the 50 degrees of freedom of both hands to 15 by eliminating irrelevant and redundant information. These approaches show that finger motions are highly redundant. We take advantage of the correlations between different degrees of freedom of the hand to optimize the capturing and animation of hand motions. 

Principal component analysis (PCA), as a standard technique to analyze and reduce high-dimensional data, has also been used to study finger motions. In Braido and Zhang's study \shortcite{BraZha04}, the two first principal components of a PCA accounted for over 98\% of all variance in the joint angles. However, their motion database did not take into account the thumb and involved only two types of tasks - cylinder grasping and voluntary flexion of individual fingers - which were repeated by different participants. Santello et al. \shortcite{SanFlaSoe98} studied a variety of 57 grasp poses and found that over 80\% of the measured 15 degrees of freedom could be described by their first two principal components. Ciocarlie...
However, all those studies apply to grasps that do not require specific motions from individual fingers. We present a method, which is optimized for American sign language (ASL), which exhibits an impressive dexterity and variety of finger motions. We hypothesize that there is less redundancy in typical finger motions of ASL than in standard grasping motions. 

One of our goals is to determine which is the most effective set of markers for capturing hand motions. Previous work has optimized marker sets for hand motions by choosing markers sets manually and a reconstructing the motions with inverse kinematics \cite{HoyRyaOSu11} or with a brute-force approach that compares the error of similar poses found in a database \cite{KanWheZor13}. 
In a similar manner, but for the full body, Chai and Hodgins, 


%Hoyet...: They find that for many cases a simple 8 marker hand model - one on each fingertip, three of the bases of the thumb, index, and little finger, produces good results. 
%Kang et al. : the best set of six markers includes the fingertips of the index, middle, ring finger as well as three markers on or near the thumb. 
Sign language: ...
Hand low-res: KanWheZor12c, CioGolAll07, ChaPolXin07

Correlations between body and finger motions, in case we manage to include data on the body: MajZorFal06, JoeHodSaf12

Body low-res?: SafHodPol04
Perception?: JoeHodOSu10


\section{Overview}

Our overall technique is divided into two stages: 1) the computation  
of the sparse marker dataset; and 2) reconstruction from the sparse 
marker set to full-resultion, skeleton-driven hand animation.
To start, we collect full-resolution motion capture 
data of hand in a small capture region. 
Specifically, our actor wears 13 small
(approx. 6mm) markers directly on the hands as well as three  
markers on the lower forearm. 
The lower forearm acts as the root link for our hand skeleton with the
assumption that these same three markers will appear in full-body captures. 
%The recorded captures act as our database. 
To account for gross body hand motion, marker positions in the database are put into the same coordinate
frame by computing the transformation of each marker relative to  
the root link.

To complete the first phase, we employ the reference data and perform PCA over the markers and 
derive a rank ordering for the markers based on their influences over
the principle components.   From this rank-ordered list, we can easily
select the top markers to act as our sparse marker set.
%
For the second phase, reconstruction, we set up a locally weighted regression 
(LWR) model to map from the sparse marker positions to a set of derived principle
components.  In this case, we conduct the PCA from the joint angles.  The LWR model is built
for each test query based on the input markers for the query sample and their proximity to 
the analogous markers in the reference data after correcting for lower arm movement.
%
Joint angles for the low dimensional input is then reconstructed 
by reversing the PCA process, from the principle components produced by 
the regression back to a full set of joint angles. 
%The joint angles are then fed back into the mapping 
%program to animate the hand.

\section{Hand Dimensionality}

At the core of our technique is the assumption that hand motion is
relatively low-dimensional.  Even though a full resolution skeleton
of the hand can have several dozen degrees of freedom (DOF),
many of the DOFs of the hand are dependent~\cite{SanFlaSoe98,BraZha04,JoeOSu09}. 
In our approach, we exploit this low dimensionality through PCA, assuming that PCA
will capture the important features of the whole-body hand
in a relatively small number of principle components. 
Further, we assume that
if we record markers that well-inform these \emph{key} principle components, we can estimate the
DOFs of the whole hand.  

To support these assumptions, we performed various tests to tease out the power of
PCA our testbed for hand animation.
Starting from a reference database, we conduct PCA to produce a new representation
of the data.  
%
%We experimented with two ways to construct 
%the joint angles from the 
%marker database.
%
Specfically, PCA is performed 
over the joint angles, giving us a new principle component 
representation of the joints angles with 54 components (assuming a skeleton of 18~joints,
with 3 Euler angles each). 
%for every sample of motion in the database.  
%
We performed an analysis to judge the ability of the PCA to directly 
reconstruct the original database motion and found that with as few as ten 
components the PCA could produce a motion with small but acceptable 
visual artifacts.  An error plot of the reconstruction error measured by the
joint angle deviation from the synthesized motion and the original motion
appears in Figure~\ref{BLAH}.  Similar findings are reported using a small
set of components from PCA to
encapsulate the motion of full-body motion in~\cite{Alla} and our results
here support similar observations made over hand data.

We also used PCA on the marker positions of the samples appearing in 
the reference database.  The dimensionality of this PCA representation
is 39, comprised of 3 root-corrected position values for each of the 13 markers.  

We support this primary assumption by showing that joint angles alone
are not capable of producing the reconstructions we realize through PCA in Figure~\ref{}. 

\begin{figure}[ht]
  \centering
  \includegraphics{images/avgError_6_3_jangles_babySigns1.jpg}
  \caption{PCA vs. Joint angles.}
\end{figure}

  components that 
sample in the database. PCA identifies the most important information 
in this database by making it the first component. Each subsequent 
component is less valuable than its predecessor. Using this knowledge, 
we sum up the values of each component over all of the samples and 
weigh each sum based on importance (which component). Each weighted 
sum correlates with a marker, and the markers are then ordered in terms 
of importance by their weighted sum. The chosen marker set is then 
used for the data we wish to reconstruct. \\

To reconstruct a sequence of motion, we first only use data from the 
marker set selected in the previous step. This is the equivalent of 
having our performer only wear these markers in the initial capture 
session. These marker position are plugged into our regression model 
to determine the full motion of our low dimensional sequence in 
principle component space. \\

%Using PCA, we can also construct an optimal marker set to represent our 
%low dimensional data. 




\section{Sparse Marker Selection} 
To construct an effective sparse marker set, our method exploits the full set of 13 markers recorded in the reference database and evaluates each marker's contribution to the whole-hand motion. 
% need description of the ref database if not given in Section 4
In constrast to the exhaustive search proposed by Kang et al.~\shortcite{KanWheZor12}, our technique computes the markers directly using PCA.

We conduct PCA with the Cartesian positions of the markers relative to the root link. With 13 markers, this leads to a PCA of motion data with 39 dimensions. In general, PCA produces a covariance matrix and the eigenvectors of this matrix create a list of components ordered from most important to least important.  
%
Each component has 39 coefficients that describe the influence of each marker on that component.  By adding up the contribution of each marker to all of the components, we rank-order the influence of the markers on the full-set of components.  Furthermore, from the eigenvalues, we are given the relative importance of each principle component with respect to each other.  We can use this importance as a weighting to bias the components.  Thus, by summing the weighted contribution of each marker to each of the components, our marker rank ordering can also account for the described bias.

%The marker coeffients are weighted depending 
%on how much the current component contributes to the final PCA reconstruction
%of the database and then summed. Markers with the highest sums are selected to
%be the sparse marker set.

In our results, we highlight sparse marker sets of three and six markers, as those form the range of what can be captured and post-processed easily based on our experience. 
Given the number of markers desired for the sparse set, we select the set simply as the top markers based on the rank-ordering.  We experimented with two methods of producing this rank-ordering, one with the eigenvalues acting as a weighting bias and the second treating all of the top-$N$ principle components as equally important and simply ignoring the remaining components. Conservatively experimenting with $N$ to be between one fourth and three fourths of the full dimensionality, these two approaches produced similar results.   However, if we selected $N$ to be the value of the full dimensionality, we see reduced quality solutions.  In practice, we employ the eigenvalue weighted ranking for all results showcased in this paper.  

A nice feature of selecting the marker set in this fashion is that the rank-ordering simply adds subsequent markers from smaller sets to produce the larger sets. Thus, the described priority ranking reveals which are the definitively \emph{most} influential markers regardless of the
size of the sparse marker set. And so, in practice, adding more markers for higher quality recordings does not require a complete change of markers, only the addition of the desired number of markers to the ones employed in the lower quality recording.

%upon the respective marker's contribution to the PCA 
%reconstruction of the reference database.








\section{Reconstruction}

A small number of markers alone cannot produce a fully
realized hand animation. Therefore, we build a regression model 
to construct joint angle measurements for a full motion sequence.
Our LWR model maps marker positions in our reference database
to joint angles of our database represented in principle
component space.

For this step, we introduce an input motion sequence
that uses the sparse marker set. The input data is 
plugged into the regression model to predict the best
joint angle components to be used for final reconstruction.
Reconstruction is completed when PCA process described in 
Section 4 is reversed, using the newly predicted components
as input. The result is a full set of joint angles for the
input motion sequence.


%We test our method on a variety sign language 
%sequences. 

%RESULTS?
% We determine the quality of these animations
% by comparing the reconstructed joint angles to ground truth
% original data. Our method also employs the principle
% component analysis (PCA), and we can access quality by 
% comparing the components of the reconstructed joint angles
% to the components of original joint angles.




\begin{figure*}[ht]
  \centering
  \includegraphics[width = 15cm] {images/marker_sets.jpg} %width=2.5in
  \caption{The tested marker sets.}
  \label{fig:marker_sets}
\end{figure*}

\section{Results}
Our primary database is used to reconstruct American Sign Language. 
The database is composed of two continuous
runs of the letters of the 
alphabet, signed by the same actor. We test the database
on various sequences that include ``word'' signs (e.g. boy or girl),
which are not included in the database.




For our sparse marker set, we choose to use six markers as our
baseline in order to compare our technique to existing solutions. 
Using the method described in Section 5 to determine 
marker importance, the markers chosen are all of the fingertips
and one on the lower part of the index finger. When we choose a
marker set of three, the markers chosen are the fingertips of the 
middle, ring, and pinky fingers (Figure~\ref{marker_sets}).

Our method uses regression to predict principle components for
a sequence of motion. In Figure~\ref{fig:BabySigns_comps}, we compare
the top three predicted components of a sequence of sign language
to the components of the original sequence with a full marker set. Plots
of the components of six markers and three markers are shown.
Though there are differences, the motion of each component closely
follows that of the ground truth for both marker sets. This can also be seen
in a reconstruced animation in the accompanying video.


\begin{figure}[h!]
  \centering
  \includegraphics[trim = 28mm 0mm 0mm 0mm,
width=0.45\textwidth]{images/Components_babySigns1.jpg} %width=2.5in
  \caption{Comparison of the components of a reconstructed clip of baby
sign language when using 6~markers and 3~markers. Ground Truth is
the original clip recorded with 13~markers.}
  \label{fig:BabySigns_comps}
\end{figure}

We compare our marker set of six to markers sets derived from the manually
selected set, proposed by Hoyet et al. (2011) and the cluster pose error
method proposed by Kang et al.(2012). Using the regression
method, our marker set produces a smaller average joint angle error per frame 
for all of our current sign language tests. Figure~\ref{fig:3_methods} shows 
these differences, again using the previous sign 
language clip as an example. The three distinct poses reached
in the sign language clip are also shown using the different marker
sets in Figure~\ref{fig:BabySigns_methods}. Our marker set is consistently
close to the original pose where as
the two other marker sets fail at achieving at least one pose.

To test the robustness of the database, we attempt to reconstruct 
motions that are not sign language. The motions we test include 
counting and general gesticulations. Our sparse marker set of six
successfully reconstructs counting the numbers~1 through~5, but the marker
set of three fails to reconstruct the number~5. For the gesture based motion,
many of the general poses in the sequence appear to be reached, but the accuracy
of the joint angles is visibly not as good the sign language reconstructions. We
then test to see if the use of another database can improve the gesture
reconstruction. Our method is performed using a gesture database. The selected
sparse marker sets of six and three are different than our previous sets.
Using the gesture database results in high quality gesture reconstructions
for both marker sets of six and three.

\begin{figure}
  \centering
  \includegraphics[trim = 28mm 0mm 0mm 0mm,
width=0.45\textwidth]{images/avgError_Marker_sets.jpg} %width=2.5in
  \caption{Comparison of three marker set selection methods that use 6 markers.}
  \label{fig:3_methods}
\end{figure}


% \begin{figure}[ht]
%   \centering
%   \includegraphics[trim = 28mm 0mm 0mm 0mm,
% width=0.45\textwidth]{images/avgError_6_3_jangles_babySigns1.jpg} %width=2.5in
%   \caption{Comparison of two regression methods: regression to principle
% components and regression to joint angles.}
%   \label{fig:PCA_noPCA}
% \end{figure}


\begin{figure}
  \centering
  \includegraphics[trim = 28mm 0mm 0mm 0mm,
width=0.45\textwidth]{images/compiled_babySigns1_poses.jpg} %width=2.5in
  \caption{The three distinct poses of the baby sign language clip
reconstructed with the three different marker sets. They are compared to
the original poses.}
  \label{fig:BabySigns_methods}
\end{figure}







\section{Discussion}

When performing the regression we map marker positions to a
certain number of components. We experimented with a smaller 
number of components would produce a better
reconstruction of the joint angle data. We found that mapping
to the full amount of components (54) produces the smallest
average error,
we can map up to 35 components with very little degradation
from a full component set.\\

%Because mapping to a full set of components produces the smallest
%average error, 
We also tested using our locally-weighted
regression model to map marker positions directly to joint angles
represented as Euler angles. The average joint angle error per
frame was very high. This can easily be seen in the animations
produced by the reconstructed joint angles. The hand does not
reach the majority of the poses in the motion. From this we see
that there is a clear benefit to using and producing principle
components to reconstruct the joint angles of the hand over 
mapping directly to joint angles.\\

We also experimented with
 three has a larger average error than the marker
set of six, but still appears to produce reasonable results. We
can see this when looking at the top principle components of the
reconstructed motion and comparing it to the top principle 
components of the original motion. For one sequence of motion,
(in Figure BLAH), the top three components for both the original
motion and the reconstructed motion with 3 markers appear to follow
very similar patterns. Although there is information lost in the
reconstruction, the general pattern of motion is the same.\\

Lastly, we attempt to reconstruct motions that are not sign language
using our alphabet database. The motions include counting and general
gesticulations. While the general poses in the 
sequences appear to be reached, the accuracy of the joint angles
is visibly not as good the sign language reconstructions. It
may be necessary to have a different database to properly reconstruct
these motions.\\
\section{Conclusion}

In this work, we present a method to capture subtle hand motions with a sparse marker set consisting of three to six markers. Our method first specifies an appropriate set of markers using principle component analysis to exploit the redundancies and irrelevancies present in hand motion data. It then reconstructs the full hand motion based on the sparse marker set with a locally weighted regression mapping marker positions to principle components. 

We show that our technique can reconstruct complex finger motions based on only three markers and outperforms methods presented by Hoyet et. al \shortcite{HoyRyaOSu11} and Kang et. al \shortcite{KanWheZor12} in recent years. Our findings also indicate that using a regression mapping marker positions to principle components leads to better results for reconstruction of the full hand motion than using a regression mapping marker positions directly to joint angles. 

The main limitation of our work is that the selection of the markers to capture is not readily applicable to other types of hand motions and the first step of our method -- computing an efficient sparse set of markers based on a database of hand motion -- has to be performed for every type of hand motions. Future work will explore what sparse marker sets would be most valuable for other types of hand motions such as grasping motions or gestures accompanying speech and thus investigate how far our approach generalizes to different types of hand motion databases.




\iffalse
\begin{equation}
 \sum_{j=1}^{z} j = \frac{z(z+1)}{2}
\end{equation}

\begin{eqnarray}
x & \ll & y_{1} + \cdots + y_{n} \\
  & \leq & z
\end{eqnarray}


\begin{figure}[ht]
  \centering
  \includegraphics[width=1.5in]{images/samplefigure}
  \caption{Sample illustration.}
\end{figure}
\fi


%\section*{Acknowledgements}

\bibliographystyle{acmsiggraph}
\bibliography{signPCA}
\end{document}