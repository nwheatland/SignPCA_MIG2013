
\section{Sparse Marker Selection}
Our method for constructing the sparse marker set is
based upon the respective marker's contribution to the PCA 
reconstruction of the reference database.
%Using PCA, we can also construct an optimal marker set to represent our 
%low dimensional data. 

To complete the first phase, we collect full-resolution motion capture 
data of hand in a small capture region. 
Specifically, our actor wears 13 small
(approx. 6mm) markers directly on the hands as well as three larger 
markers on the lower forearm. 
The lower forearm acts as the root of the hand. The recorded 
captures act as our database. Marker positions in the database 
are placed in the same coordinate
frame by determining the offset of each marker from the root body.

%PUT THIS IS RESULTS?
%Our database contains two versions of the 
%alphabet signed by the same person and has 6750 frames. The other 
%captures are used as input to our system to be fully reconstructed. \\


The second phase of this process is performing PCA on the database.
PCA produces a covariance matrix and the eigenvectors
of this matrix create a list of components, ordered from most important
to least important.

Each component has 39 dimensions, where each triplet represents the 
\emph{xyz} coefficients of a marker and that marker's contribution
to the current component. The marker coefficients are weighted depending 
on how much the current component contributes to the final PCA reconstruction
of the database and then summed. Markers with the highest sums are selected to
be the sparse marker set.


