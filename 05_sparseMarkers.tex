
\section{Sparse Marker Selection} 
To construct an effective sparse marker set, our method exploits the full set of 13 markers recorded in the reference database and evaluates each marker's contribution to the whole-hand motion. 
% need description of the ref database if not given in Section 4
In constrast to the exhaustive search proposed by Kang et al.~\shortcite{KanWheZor12}, our technique computes the markers directly using PCA.

We conduct PCA with the Cartesian positions of the markers relative to the root link. With 13 markers, this leads to a PCA of motion data with 39 dimensions. In general, PCA produces a covariance matrix and the eigenvectors of this matrix create a list of components ordered from most important to least important.  
%
Each component has 39 coefficients that describe the influence of each marker on that component.  By adding up the contribution of each marker to all of the components, we rank-order the influence of the markers on the full-set of components.  Furthermore, from the eigenvalues, we are given the relative importance of each principle component with respect to each other.  We can use this importance as a weighting to bias the components.  Thus, by summing the weighted contribution of each marker to each of the components, our marker rank ordering can also account for the described bias.

%The marker coeffients are weighted depending 
%on how much the current component contributes to the final PCA reconstruction
%of the database and then summed. Markers with the highest sums are selected to
%be the sparse marker set.

In our results, we highlight sparse marker sets of three and six markers, as those form the range of what can be captured and post-processed easily based on our experience. 
Given the number of markers desired for the sparse set, we select the set simply as the top markers based on the rank-ordering.  We experimented with two methods of producing this rank-ordering, one with the eigenvalues acting as a weighting bias and the second treating all of the top-$N$ principle components as equally important and simply ignoring the remaining components. Conservatively experimenting with $N$ to be between one fourth and three fourths of the full dimensionality, these two approaches produced similar results.   However, if we selected $N$ to be the value of the full dimensionality, we see reduced quality solutions.  In practice, we employ the eigenvalue weighted ranking for all results showcased in this paper.  

A nice feature of selecting the marker set in this fashion is that the rank-ordering simply adds subsequent markers from smaller sets to produce the larger sets. Thus, the described priority ranking reveals which are the definitively \emph{most} influential markers regardless of the
size of the sparse marker set. And so, in practice, adding more markers for higher quality recordings does not require a complete change of markers, only the addition of the desired number of markers to the ones employed in the lower quality recording.

%upon the respective marker's contribution to the PCA 
%reconstruction of the reference database.






